% Options for packages loaded elsewhere
\PassOptionsToPackage{unicode}{hyperref}
\PassOptionsToPackage{hyphens}{url}
\PassOptionsToPackage{dvipsnames,svgnames,x11names}{xcolor}
%
\documentclass[
  letterpaper,
  DIV=11,
  numbers=noendperiod]{scrartcl}

\usepackage{amsmath,amssymb}
\usepackage{iftex}
\ifPDFTeX
  \usepackage[T1]{fontenc}
  \usepackage[utf8]{inputenc}
  \usepackage{textcomp} % provide euro and other symbols
\else % if luatex or xetex
  \usepackage{unicode-math}
  \defaultfontfeatures{Scale=MatchLowercase}
  \defaultfontfeatures[\rmfamily]{Ligatures=TeX,Scale=1}
\fi
\usepackage{lmodern}
\ifPDFTeX\else  
    % xetex/luatex font selection
\fi
% Use upquote if available, for straight quotes in verbatim environments
\IfFileExists{upquote.sty}{\usepackage{upquote}}{}
\IfFileExists{microtype.sty}{% use microtype if available
  \usepackage[]{microtype}
  \UseMicrotypeSet[protrusion]{basicmath} % disable protrusion for tt fonts
}{}
\makeatletter
\@ifundefined{KOMAClassName}{% if non-KOMA class
  \IfFileExists{parskip.sty}{%
    \usepackage{parskip}
  }{% else
    \setlength{\parindent}{0pt}
    \setlength{\parskip}{6pt plus 2pt minus 1pt}}
}{% if KOMA class
  \KOMAoptions{parskip=half}}
\makeatother
\usepackage{xcolor}
\ifLuaTeX
  \usepackage{luacolor}
  \usepackage[soul]{lua-ul}
\else
  \usepackage{soul}
  
\fi
\setlength{\emergencystretch}{3em} % prevent overfull lines
\setcounter{secnumdepth}{5}
% Make \paragraph and \subparagraph free-standing
\ifx\paragraph\undefined\else
  \let\oldparagraph\paragraph
  \renewcommand{\paragraph}[1]{\oldparagraph{#1}\mbox{}}
\fi
\ifx\subparagraph\undefined\else
  \let\oldsubparagraph\subparagraph
  \renewcommand{\subparagraph}[1]{\oldsubparagraph{#1}\mbox{}}
\fi


\providecommand{\tightlist}{%
  \setlength{\itemsep}{0pt}\setlength{\parskip}{0pt}}\usepackage{longtable,booktabs,array}
\usepackage{calc} % for calculating minipage widths
% Correct order of tables after \paragraph or \subparagraph
\usepackage{etoolbox}
\makeatletter
\patchcmd\longtable{\par}{\if@noskipsec\mbox{}\fi\par}{}{}
\makeatother
% Allow footnotes in longtable head/foot
\IfFileExists{footnotehyper.sty}{\usepackage{footnotehyper}}{\usepackage{footnote}}
\makesavenoteenv{longtable}
\usepackage{graphicx}
\makeatletter
\def\maxwidth{\ifdim\Gin@nat@width>\linewidth\linewidth\else\Gin@nat@width\fi}
\def\maxheight{\ifdim\Gin@nat@height>\textheight\textheight\else\Gin@nat@height\fi}
\makeatother
% Scale images if necessary, so that they will not overflow the page
% margins by default, and it is still possible to overwrite the defaults
% using explicit options in \includegraphics[width, height, ...]{}
\setkeys{Gin}{width=\maxwidth,height=\maxheight,keepaspectratio}
% Set default figure placement to htbp
\makeatletter
\def\fps@figure{htbp}
\makeatother

\KOMAoption{captions}{tableheading}
\makeatletter
\@ifpackageloaded{caption}{}{\usepackage{caption}}
\AtBeginDocument{%
\ifdefined\contentsname
  \renewcommand*\contentsname{Table of contents}
\else
  \newcommand\contentsname{Table of contents}
\fi
\ifdefined\listfigurename
  \renewcommand*\listfigurename{List of Figures}
\else
  \newcommand\listfigurename{List of Figures}
\fi
\ifdefined\listtablename
  \renewcommand*\listtablename{List of Tables}
\else
  \newcommand\listtablename{List of Tables}
\fi
\ifdefined\figurename
  \renewcommand*\figurename{Figure}
\else
  \newcommand\figurename{Figure}
\fi
\ifdefined\tablename
  \renewcommand*\tablename{Table}
\else
  \newcommand\tablename{Table}
\fi
}
\@ifpackageloaded{float}{}{\usepackage{float}}
\floatstyle{ruled}
\@ifundefined{c@chapter}{\newfloat{codelisting}{h}{lop}}{\newfloat{codelisting}{h}{lop}[chapter]}
\floatname{codelisting}{Listing}
\newcommand*\listoflistings{\listof{codelisting}{List of Listings}}
\makeatother
\makeatletter
\makeatother
\makeatletter
\@ifpackageloaded{caption}{}{\usepackage{caption}}
\@ifpackageloaded{subcaption}{}{\usepackage{subcaption}}
\makeatother
\ifLuaTeX
  \usepackage{selnolig}  % disable illegal ligatures
\fi
\usepackage{bookmark}

\IfFileExists{xurl.sty}{\usepackage{xurl}}{} % add URL line breaks if available
\urlstyle{same} % disable monospaced font for URLs
\hypersetup{
  pdftitle={Household Assets and Fuelwood Use},
  pdfauthor={Renato Vargas; Natsuko Kiso Nozaki},
  colorlinks=true,
  linkcolor={blue},
  filecolor={Maroon},
  citecolor={Blue},
  urlcolor={Blue},
  pdfcreator={LaTeX via pandoc}}

\title{Household Assets and Fuelwood Use}
\author{Renato Vargas \and Natsuko Kiso Nozaki}
\date{}

\begin{document}
\maketitle

\renewcommand*\contentsname{Table of contents}
{
\hypersetup{linkcolor=}
\setcounter{tocdepth}{3}
\tableofcontents
}
\section{Household assets}\label{household-assets}

\textbf{Homes are perhaps the most important asset owned by households.
In 2022,} \textbf{most of the 800,604 Armenian households owned their
dwelling (about 89.4\%)}, as shown in Figure 1, with only about 5.9\% of
homes renting and 4.7\% with other forms of tenure (ARMSTAT, 2023).
Also, about 97\% of households lived in houses or apartments, as opposed
to hostel; railcar / container; other temporary lodging; or ``other''
(3\% combined). Houses averaged 112.0 m\textsuperscript{2}, while
apartments averaged 68.3 m\textsuperscript{2}. As expected, 96.2\% of
apartments were located in urban areas, with about half of them in
Yerevan (55.3\%). Houses, on the other hand, were located mostly in
rural areas (65.8\%), with about 13.4\% of them located in Yerevan.

\phantomsection\label{_Ref100306330}{}Figure 1. Dwelling ownership by
Marz

{{[}CHART{]}}

Source: Integrated Living Conditions Survey, 2022 (ARMSTAT, 2023).

\textbf{The rental market is small and an urban phenomenon, with 92\% of
rentals occurring in that area.} The average rent for a house was
AMD~56,613.6 (about USD~143.1) with an average price of AMD~882.4
(USD~2.23) per square meter. Conversely, apartments were rented at a
more expensive mean of AMD~82,124.2 (USD~207.6) with an average price of
AMD~1,447.8 (USD~3.7) per square meter.

\textbf{Owned dwellings are an asset from which households derive
welfare. Non renters derived an average of AMD~54,338.1 (USD~137.33) in
monthly imputed rent.} The emergent rental market information was used
to impute rent to non-renters using a log linear modeling approach
described by Ceriani et al.~(2019), in which imputed rent was predicted
using a combination of household characteristics (urban/rural, Marz,
number of rooms, presence of an indoor toilet, number of household,
square meters, type of dwelling, members) and head of household
characteristics (sex, highest completed schooling level, age group).
These values are shown in Table 1 by decile and for the whole country.
Net present value of that monthly imputed rent (for a 2050 horizon) was
also estimated at AMD~17.3~million (USD~43,881.5), using a 5\% annual
discount rate and a 5\% average inflation rate.

\phantomsection\label{_Ref100310659}{}Table 1. Imputed rent and average
net present value (2050 horizon) for non-renters

\begin{longtable}[]{@{}llll@{}}
\toprule\noalign{}
Decile & Average dwelling area (m\textsuperscript{2}) & Average imputed
rent (Dram per month) & Average net present value of rent (2050
horizon) \\
\midrule\noalign{}
\endhead
\bottomrule\noalign{}
\endlastfoot
1 & 89.4 & 45,632.3 & 14,565,638.7 \\
2 & 91.1 & 51,149.9 & 16,326,813.0 \\
3 & 91.1 & 52,635.7 & 16,801,095.0 \\
4 & 89.5 & 55,010.5 & 17,559,108.4 \\
5 & 88.3 & 53,896.9 & 17,203,666.3 \\
6 & 87.2 & 54,230.5 & 17,310,143.3 \\
7 & 93.0 & 54,532.6 & 17,406,560.2 \\
8 & 89.1 & 52,558.3 & 16,776,366.9 \\
9 & 86.7 & 56,791.0 & 18,127,439.9 \\
10 & 87.5 & 59,735.1 & 19,067,188.6 \\
Country & 89.0 & 54,338.1 & 17,344,482.6 \\
\end{longtable}

Source: author based on log linear imputed rent approach (Ceriani et
al., 2019)

\textbf{A little over a third (288,718 or 36.1\%) of households owned a
car in 2022 and used it in the month prior to the survey.} However, a
higher percentage of homes in rural areas (47.5\%) own a vehicle. Given
the characteristics of rural areas and the availability of public
transportation, it is a particularly important asset for household
mobility. This also means that these households are exposed to energy
transition risks derived from changes to prices of fuels and
technological changes in car technologies (TNFD, 2023).

\textbf{Armenians access water mainly through centralized water supply
(95.6\% or 765,728 households)}, 2.7\% of households have their own
system of water supply, and the remaining 1.6\% access water through
spring water or well; delivered water; bought water; or ``other''. Urban
households spend an average of about AMD~2,506.5 (USD~6.3) on water,
while rural households spend about AMD~1,790.1 (USD~4.5).

\textbf{Most homes (99.8\% or 798,835 households) access electricity
through the national grid, with only a small share 0.2\% using solar
panels.} Both urban and rural homes spend about the same average
expenditure on electricity of about AMD~7,948.3 (USD~20.1).

\section{Household exposure to the agricultural
sector}\label{household-exposure-to-the-agricultural-sector}

\textbf{About a fourth of all Armenian households (23.4\%) had a monthly
agricultural income component; a number that rises to 61\% when
discussing rural households, with that income representing an average
22.6\% of total income.} Overall, 91.3\% of those 187,176 households
that derived an agricultural income were located in rural areas. This
income averaged AMD~72,275.2 (USD~182.67) and represented an average of
23.7\% of total income for rural homes. These same figures averaged
AMD~47,853.1 (USD~120.94) and 12.0\% in urban areas, respectively. When
it comes to deciles, perhaps unintuitively, the average share of total
income that comes from agriculture rises from 20.7\% for the first
decile to 26.3\% for the tenth, as shown in Table 2.

\phantomsection\label{_Ref154750198}{}Table 2. Average agricultural
income

\begin{longtable}[]{@{}
  >{\raggedright\arraybackslash}p{(\columnwidth - 6\tabcolsep) * \real{0.1235}}
  >{\raggedright\arraybackslash}p{(\columnwidth - 6\tabcolsep) * \real{0.1975}}
  >{\raggedright\arraybackslash}p{(\columnwidth - 6\tabcolsep) * \real{0.2840}}
  >{\raggedright\arraybackslash}p{(\columnwidth - 6\tabcolsep) * \real{0.3704}}@{}}
\toprule\noalign{}
\begin{minipage}[b]{\linewidth}\raggedright
Decile
\end{minipage} & \begin{minipage}[b]{\linewidth}\raggedright
Average total

income
\end{minipage} & \begin{minipage}[b]{\linewidth}\raggedright
Average agricultural

Income
\end{minipage} & \begin{minipage}[b]{\linewidth}\raggedright
Average agricultural income

share of total income
\end{minipage} \\
\midrule\noalign{}
\endhead
\bottomrule\noalign{}
\endlastfoot
1 & 272,739.6 & 54,013.3 & 20.7\% \\
2 & 277,204.7 & 47,470.2 & 19.3\% \\
3 & 308,608.9 & 75,184.5 & 22.0\% \\
4 & 280,508.0 & 51,349.7 & 18.3\% \\
5 & 306,497.2 & 68,414.4 & 24.0\% \\
6 & 301,499.4 & 69,378.7 & 22.8\% \\
7 & 315,422.2 & 72,267.4 & 21.5\% \\
8 & 337,909.6 & 78,487.8 & 25.5\% \\
9 & 295,644.2 & 78,357.8 & 23.5\% \\
10 & 315,725.2 & 92,134.7 & 26.3\% \\
Country & 303,234.5 & 70,156.4 & 22.6\% \\
\end{longtable}

Source: author based on Integrated Living Conditions Survey, 2022
(ARMSTAT, 2023).

Note: column three is the average of the share calculated at the
household level and weighted by population weights, not column two
divided by column one.

\textbf{While only 23.4\% of households, derive income from the
agricultural sector, exposure to agriculture is larger, since 41\% of
households use land for agricultural purposes (owned and/or rented).} A
total of 98\% out of the 328,438 households that use agricultural land
own their plots, and 5.3\% of those (also) rented. The average area used
by households for agricultural purposes is 7,329.9~m\textsuperscript{2},
of which an average 70.6\%, or 5,551.2~m\textsuperscript{2}, is used for
crops. This suggests that own consumption of agricultural output plays a
role in Armenian incomes. Table 3 shows that for households without
agricultural sales, 4.3\% of income can be attributed to imputed use of
agricultural products for own consumption. More generally, this table
shows the shares of household income from different sources, for
households with agricultural land, with or without deriving income from
that land, compared with households without agricultural land. It is
evident that public pensions and benefits plays a much bigger role
(29.7\%) for all households, along with hired employment (41.3\%).

\phantomsection\label{_Ref156594885}{}Table 3. Average shares of sources
of income for households with and without agricultural land

\begin{longtable}[]{@{}
  >{\raggedright\arraybackslash}p{(\columnwidth - 8\tabcolsep) * \real{0.1918}}
  >{\raggedright\arraybackslash}p{(\columnwidth - 8\tabcolsep) * \real{0.1918}}
  >{\raggedright\arraybackslash}p{(\columnwidth - 8\tabcolsep) * \real{0.1918}}
  >{\raggedright\arraybackslash}p{(\columnwidth - 8\tabcolsep) * \real{0.1918}}
  >{\raggedright\arraybackslash}p{(\columnwidth - 8\tabcolsep) * \real{0.1918}}@{}}
\toprule\noalign{}
\begin{minipage}[b]{\linewidth}\raggedright
\end{minipage} & \begin{minipage}[b]{\linewidth}\raggedright
Households with a gricultural land (owned or not) with a gricultural
income
\end{minipage} & \begin{minipage}[b]{\linewidth}\raggedright
Households with a gricultural land (owned or not) with no a gricultural
income
\end{minipage} & \begin{minipage}[b]{\linewidth}\raggedright
Households without a gricultural land (owned or not) with or without a
gricultural income
\end{minipage} & \begin{minipage}[b]{\linewidth}\raggedright
All households
\end{minipage} \\
\midrule\noalign{}
\endhead
\bottomrule\noalign{}
\endlastfoot
Number of households & 184,738 & 143,700 & 472,166 & 800,604 \\
Average total income (Dram) & 302,290.3 & 223,874.8 & 247,434.9 &
255,863.9 \\
Average share of income coming from: & & & & \\
\begin{minipage}[t]{\linewidth}\raggedright
\begin{quote}
Sale of a gricultural products
\end{quote}
\end{minipage} & 22.8\% & 0.0\% & 0.1\% & 5.3\% \\
\begin{minipage}[t]{\linewidth}\raggedright
\begin{quote}
Imputed use of a gricultural products for own

consumption
\end{quote}
\end{minipage} & 8.7\% & 4.3\% & 0.6\% & 3.2\% \\
\begin{minipage}[t]{\linewidth}\raggedright
\begin{quote}
Hired

employment
\end{quote}
\end{minipage} & 28.5\% & 40.5\% & 46.6\% & 41.3\% \\
\begin{minipage}[t]{\linewidth}\raggedright
\begin{quote}
Self -employment
\end{quote}
\end{minipage} & 10.2\% & 9.2\% & 8.0\% & 8.7\% \\
\begin{minipage}[t]{\linewidth}\raggedright
\begin{quote}
Property (rent, interest, equity gain)
\end{quote}
\end{minipage} & 0.1\% & 0.1\% & 0.2\% & 0.2\% \\
\begin{minipage}[t]{\linewidth}\raggedright
\begin{quote}
Public pensions and benefits
\end{quote}
\end{minipage} & 22.2\% & 33.8\% & 31.5\% & 29.7\% \\
\begin{minipage}[t]{\linewidth}\raggedright
\begin{quote}
Transfers
\end{quote}
\end{minipage} & 7.2\% & 10.5\% & 11.6\% & 10.4\% \\
\begin{minipage}[t]{\linewidth}\raggedright
\begin{quote}
Other
\end{quote}
\end{minipage} & 0.7\% & 1.3\% & 1.4\% & 1.2\% \\
All income shares & 100\% & 100\% & 100\% & 100\% \\
\end{longtable}

Source: author based on Integrated Living Conditions Survey, 2022
(ARMSTAT, 2023).

\section{Household reliance on firewood for
heating}\label{household-reliance-on-firewood-for-heating}

\subsection{Expenditure elasticities for
fuelwood}\label{expenditure-elasticities-for-fuelwood}

\textbf{Most homes in Armenia use natural gas for heating (61.9\% or
495,203 households), but an important 23.8\%~(190,884 households) use
wood for heating, followed by 21.7\% that use electricity and 8.2\%
pressed dung.} Negligible percentages of households use liquefied gas or
coal (0.2\% and 0.9\% respectively). In rural areas, 51.5\% of rural
homes (143,724 households) use wood for heating, which correlates with
the 54.8\% of rural households that have a self-made heater as main
technology. Not only is wood used as heating source, but it is also used
as an ``energy carrier'' (mainly cooking) by 11.3\% of all households.
In rural areas, 25.8\% of homes use it for this purpose.

\textbf{While there is a market for fuelwood, many homes do not pay for
fuelwood annually.} Figure 2 shows the distribution for this concept,
showing a clear component of households paying zero Dram for their
fuelwood consumption annually. This is possibly related to those
households that collect wood for free. However, as shown in Table 4,
homes that do pay for fuelwood spend an average of AMD~128,209.2 (about
USD~317.8) annually, which is roughly 5.5\% of total annual expenditure
for the average household.

\phantomsection\label{_Ref154834559}{}Figure 2. Annual wood expenditure
distribution

\includegraphics[width=6.28286in,height=3.79769in]{media/image1.png}

Source: Integrated Living Conditions Survey, 2022 (ARMSTAT, 2023).

\phantomsection\label{_Ref154909808}{}Table 4. Average fuelwood
expenditure and quantity used annually and in the month prior to the
survey

\begin{longtable}[]{@{}
  >{\raggedright\arraybackslash}p{(\columnwidth - 12\tabcolsep) * \real{0.0334}}
  >{\raggedright\arraybackslash}p{(\columnwidth - 12\tabcolsep) * \real{0.1773}}
  >{\raggedright\arraybackslash}p{(\columnwidth - 12\tabcolsep) * \real{0.1806}}
  >{\raggedright\arraybackslash}p{(\columnwidth - 12\tabcolsep) * \real{0.1839}}
  >{\raggedright\arraybackslash}p{(\columnwidth - 12\tabcolsep) * \real{0.1906}}
  >{\raggedright\arraybackslash}p{(\columnwidth - 12\tabcolsep) * \real{0.1104}}
  >{\raggedright\arraybackslash}p{(\columnwidth - 12\tabcolsep) * \real{0.1171}}@{}}
\toprule\noalign{}
\begin{minipage}[b]{\linewidth}\raggedright
Area
\end{minipage} & \begin{minipage}[b]{\linewidth}\raggedright
Average monthly household total expenditure (Dram)
\end{minipage} & \begin{minipage}[b]{\linewidth}\raggedright
Average monthly household energy expenditure (Dram)
\end{minipage} & \begin{minipage}[b]{\linewidth}\raggedright
Fuelwood expenditure in month prior to survey (Dram)
\end{minipage} & \begin{minipage}[b]{\linewidth}\raggedright
Fuelwood quantity used in month prior to survey (m\textsuperscript{3})
\end{minipage} & \begin{minipage}[b]{\linewidth}\raggedright
Fuelwood expenditure annually\\
(Dram)\strut
\end{minipage} & \begin{minipage}[b]{\linewidth}\raggedright
Fuelwood quantity used annually\\
(m\textsuperscript{3})\strut
\end{minipage} \\
\midrule\noalign{}
\endhead
\bottomrule\noalign{}
\endlastfoot
URBAN & 187,752.5 & 12,700.4 & 2,510.2 & 0.6 & 116,428.6 & 6.4 \\
RURAL & 201,881.6 & 15,652.8 & 1,821.9 & 0.6 & 131,653.5 & 6.8 \\
Country & 192,673.8 & 13,728.7 & 1,977.6 & 0.6 & 128,209.2 & 6.7 \\
\end{longtable}

Source: author based on Integrated Living Conditions Survey, 2022
(ARMSTAT, 2023).

However, an \textbf{expenditure elasticity of annual wood expenditure
across income deciles}\footnote{This is calculated as: Decile
  Expenditure Elasticity of Demand = \% Change in average expenditure
  from decile to decile / \% Change in average Fuelwood Expenditure
  Demanded by (expenditure) decile.

  \hspace{0pt}} \textbf{(when moving from decile to decile)} points to a
complex relationship between income and fuelwood expenditure. The higher
elasticities in the middle deciles might reflect an increased ability
and desire to spend on fuelwood, possibly for heating during colder
months. The negative figures in the higher deciles could suggest a shift
towards more modern heating solutions, or that beyond a certain income
level, the relative importance of fuelwood decreases. The variations
between deciles underscore the diverse factors influencing energy
choices, including affordability, accessibility, and preferences. (see
Table 5). \st{The higher expenditure elasticity for fuelwood in decile 5
could be indicative of larger household sizes or larger homes that
require more wood for heating and cooking}. This would naturally lead to
a higher quantity of fuelwood being used, which would increase
expenditure on fuelwood without necessarily implying luxury consumption
that theory would suggest with high elasticities.

This interpretation would be consistent with a scenario where these
households, perhaps due to their size or location, have not yet
transitioned to other energy sources, which are often more accessible in
urban settings or to wealthier households that can afford the initial
investment in more modern heating systems. This considers the
socio-economic context that might prevent a switch to alternative fuels,
even when households have slightly more income. Issues such as
availability of infrastructure, initial costs of switching to gas or
electric heating, and cultural preferences for wood.

\textbf{In rural areas for example, the availability of fuelwood and the
lack of infrastructure for alternative energy sources can lead to a
situation where households spend more on wood simply because
it\textquotesingle s one of the few options available to them, and not
because they particularly prefer wood over other fuels.} Indeed, when
conducting this analysis across urban to rural, the elasticity value
reaches 3.25 (growth of wood expenditure is three times the growth of
total consumption across urban/rural), confirming that living in rural
areas might be a big determinant of fuelwood use, simply because it's
the technology to which those homes have access and other infrastructure
is lacking.

\phantomsection\label{_Ref154862129}{}Table 5. Expenditure elasticity of
annual wood expenditure across deciles

\begin{longtable}[]{@{}ll@{}}
\toprule\noalign{}
Decile & Elasticity of Annual Wood Expenditure \\
\midrule\noalign{}
\endhead
\bottomrule\noalign{}
\endlastfoot
1 & - \\
2 & 0.5 \\
3 & 0.2 \\
4 & 1.6 \\
5 & 3.3 \\
6 & 2.1 \\
7 & -0.3 \\
8 & 0.4 \\
9 & -0.1 \\
10 & -0.2 \\
\end{longtable}

Source: author based on Integrated Living Conditions Survey, 2022
(ARMSTAT, 2023).

For comparison, the expenditure elasticity of natural gas expenditure
(which has a more traditionally priced market) across income deciles
suggest that natural gas is not consistently treated as a normal or
inferior good across the income spectrum. The negative elasticities in
higher deciles may indicate a trend of higher-income households either
becoming more energy-efficient, having better-insulated homes, or
switching to alternative fuels that are perceived as cleaner or more
convenient. This might be consistent with the higher percentage of urban
homes that use electricity for cooking and heating. The very high
negative elasticity in decile 9 is particularly notable and could
warrant a closer investigation to understand the underlying causes.

\phantomsection\label{_Toc156819377}{}Table 6. Expenditure elasticity of
natural gas expenditure across deciles

\begin{longtable}[]{@{}ll@{}}
\toprule\noalign{}
Decile & Elasticity \\
\midrule\noalign{}
\endhead
\bottomrule\noalign{}
\endlastfoot
1 & - \\
2 & 0.62 \\
3 & 0.10 \\
4 & -0.88 \\
5 & -0.44 \\
6 & 0.22 \\
7 & -0.29 \\
8 & -0.48 \\
9 & -6.59 \\
10 & -0.55 \\
\end{longtable}

Source: author based on Integrated Living Conditions Survey, 2022
(ARMSTAT, 2023).

When comparing pressed dung, the elasticity behavior shows a component
that responds much more to income levels. Table 7 shows the household
expenditure elasticity of press dung expenditure across deciles,
suggesting big leaps to lower levels of expenditure on pressed dung as
households move from decile to decile, especially moving from decile 3
to 4, from decile 5 to 6 and then from 6 to 7.

\phantomsection\label{_Ref156819358}{}Table 7. Expenditure elasticity of
pressed dung expenditure across deciles

\begin{longtable}[]{@{}ll@{}}
\toprule\noalign{}
Decile & Elasticity of Pressed Dung Expenditure \\
\midrule\noalign{}
\endhead
\bottomrule\noalign{}
\endlastfoot
1 & 0.00 \\
2 & 0.43 \\
3 & -0.56 \\
4 & -6.41 \\
5 & -3.35 \\
6 & -25.71 \\
7 & -12.18 \\
8 & 1.02 \\
9 & 0.48 \\
10 & -4.81 \\
\end{longtable}

Source: author based on Integrated Living Conditions Survey, 2022
(ARMSTAT, 2023).

\subsection{Determinants of fuelwood
consumption}\label{determinants-of-fuelwood-consumption}

There are various factors that influence fuelwood consumption by
households. While income is important, it's not the only determinant,
and there are other factors, which include affordability and
availability of fuels, scarcity of fuelwood supply, fuel preferences,
and cost and performance of end-use equipment. Moreover, all these
factors also on whether the household is in an urban or rural setting
(Lefevre et al., 1997).

To better understand this, a linear model was conducted on household
data for Armenia (to explain annual wood quantity used by household as
determined by Marz, urban/rural, total household income, dwelling size
in square meters, hectares of forests per thousand inhabitants by
Marz\footnote{Author calculated this variable using land use land cover
  information from ESA CCI Land Cover time-series v.2.0.7, 1992-2020
  (ESA, 2023) and population data (ARMSTAT, 2020).} (see Table 8),
average price of natural gas in Marz and area (urban/rural), computed as
the average of the division of total spent on natural gas by amount of
natural gas used by households in that Marz and area (urban/rural); and
average electricity bill in that Marz and Area.

The explanatory power of the model is moderate\footnote{Moderate but
  robust. The model has been probed using the Variance Inflation Factor,
  showing that the model does not appear to suffer from sever
  multicollinearity. Correlations among variables are moderate or low as
  well. A concern about interaction between urban/rural and dwelling
  size, which removed the slight significance of a previous model
  without dwelling size was ruled out, testing a model with the
  interaction of these variables, which was not significant.} (Adjusted
R-squared: 0.2668), but the results are interesting as shown in Table 9.
The baseline level of annual wood consumption is high (with an intercept
26.51 cubic meters of wood). The effect of administrative division is
small and statistically insignificant (ADM1 administrative division
-0.0271). The effect of being in a rural area, compared to an urban
area, is not significant. Results indicate that for each unit increase
in total income, wood consumption increases by approximately 0.0000029
cubic meters per year. This effect is highly significant (p \textless{}
2.26e-16), suggesting a positive relationship between income and wood
consumption, which is counterintuitive. More in line with expectations
is dwelling size, which is also highly significant, suggesting that
higher dwelling sizes lead to higher fuelwood use. The number of
hectares of forest by thousand inhabitants is a significant predictor.
The coefficient indicates that more forest availability leads to higher
wood consumption. Both the price of natural gas and average expenditure
on electricity in Marz and area are non-significant.

\phantomsection\label{_Ref154856082}{}Table 8. Hectares of forest per
thousand inhabitants

\begin{longtable}[]{@{}llllll@{}}
\toprule\noalign{}
Marz & Forested areas (Ha) & Not Forested areas (Ha) & Total area (Ha) &
Population 2020 (thousand inhabitants) & Hectares of forest per thousand
inhabitants \\
\midrule\noalign{}
\endhead
\bottomrule\noalign{}
\endlastfoot
Yerevan & - & 23,246.8 & 23,246.8 & 1,084.0 & - \\
Aragatsotn & 1,948.2 & 270,744.9 & 272,693.1 & 26.8 & 72.7 \\
Ararat & 3,417.5 & 207,130.8 & 210,548.3 & 72.1 & 47.4 \\
Armavir & 196.5 & 125,429.7 & 125,626.2 & 82.4 & 2.4 \\
Gegharkunik & 13,525.9 & 383,155.5 & 396,681.4 & 66.6 & 203.1 \\
Lori & 81,492.8 & 293,208.2 & 374,701.0 & 126.1 & 646.3 \\
Kotayk & 11,571.6 & 199,289.4 & 210,861.0 & 136.8 & 84.6 \\
Shirak & 882.2 & 269,527.6 & 270,409.8 & 135.6 & 6.5 \\
Syunik & 44,390.9 & 402,134.9 & 446,525.9 & 93.2 & 476.3 \\
Vayots Dzor & 2,500.0 & 226,086.5 & 228,586.5 & 17.1 & 146.2 \\
Tavush & 121,962.8 & 149,079.3 & 271,042.2 & 93.2 & 1,308.6 \\
\end{longtable}

Source: author land use land cover information from ESA CCI Land Cover
time-series v.2.0.7, 1992-2020 (ESA, 2023) and population data (ARMSTAT,
2020).

\phantomsection\label{_Ref154854194}{}Table 9. Regression results on the
determinants of fuelwood use

\begin{longtable}[]{@{}llllll@{}}
\toprule\noalign{}
& Min & 1Q & Median & 3Q & Max \\
\midrule\noalign{}
\endhead
\bottomrule\noalign{}
\endlastfoot
Residuals: & -13.625 & -1.716 & -0.501 & 1.35 & 13.445 \\
\end{longtable}

Coefficients: Estimate Std. Error t value
Pr(\textgreater\textbar t\textbar)\\
(Intercept) 6.19E+00 1.58E+01 0.391 0.696 ~ ADM1 administrative
divisions -2.71E-02 3.33E-02 -0.815 0.415\\
Urban/Rural 7.22E-02 2.02E-01 0.358 0.72\\
Dwelling size (m\textsuperscript{2}) 2.08E-02 2.22E-03 9.4 \textless{}
2e-16 *** Total income (Dram) 2.22E-06 3.14E-07 7.081 2.26E-12 *** Ha.
of forest / 1000 inhabitants 3.28E-03 3.23E-04 10.145 \textless{} 2e-16
*** Avg. price of natural gas (Dram/m\textsuperscript{3}) -2.47E-02
1.14E-01 -0.217 0.828\\
Avg. electricity expenditure (Dram) -2.88E-06 9.76E-05 -0.03 0.976 ~

Significance codes: 0 `***' 0.001 `**' 0.01 `*' 0.05 `.' 0.1 ' ' 1

Residual standard error: 2.808 on 1392 degrees of freedom (3784
observations deleted due to missingness)

Multiple R-squared: 0.2704, Adjusted R-squared: 0.2668.

F-statistic: 73.71 on 7 and 1392 DF, p-value: \textless{} 2.2e-16.

Overall, it can be concluded that there is an income effect that is
counter intuitive. Higher income is associated with increased wood
consumption. Wealthier households probably consume more wood due to
larger homes as captured in the effect of dwelling size, and a strong
cultural preference for wood as a fuel source with probably limited
alternatives due to infrastructure or other reasons. Forest availability
is also a strong predictor, suggesting that more available forest
resources lead to higher wood consumption. There\textquotesingle s a
tendency for wood consumption to decrease as natural gas prices
increase, though this is almost insignificant and general electricity
expenditure does not seem to be a significant determinant of wood
consumption at all.

\section{References}\label{references}

ARMSTAT. (2020). \emph{Marzes of the Republic of Armenia and Yerevan
city in figures, 2020}.

ARMSTAT. (2023). \emph{Integrated Living Conditions Survey 2022}
{[}dataset{]}.

Ceriani, L., Olivieri, S., \& Ranzani, M. (2019). Housing, imputed rent,
and households' welfare. \emph{Poverty \& Equity Global Practice Working
Papers}, \emph{213}.
https://documents1.worldbank.org/curated/pt/336451565194643402/pdf/Housing-Imputed-Rent-and-Households-Welfare.pdf

ESA. (2023). \emph{ESA CCI Land Cover time-series v.2.0.7 (1992-2020)}.
European Space Agency.

Lefevre, T., Todoc, J., \& Raj Timilsina, G. (1997). \emph{The Role of
Wood Energy in Asia}. Food and Agriculture Organization of the United
Nations.

TNFD. (2023). \emph{Recommendations of the Taskforce on Nature-related
Financial Disclosures}. Green Finance Institute.



\end{document}
